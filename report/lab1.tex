\documentclass{article}
\newcommand{\upcite}[1]{\textsuperscript{\textsuperscript{\cite{#1}}}}

% Packages
\usepackage{titlesec}
\usepackage{indentfirst}
\usepackage{amsmath, amsthm, amssymb, graphicx}
\usepackage{graphicx}

% Title
\title{Satellite Lab1}
\author{Group6: Zhengyang Hua, Xipeng Li, Yushuo Feng}
\date{\today}

\begin{document}

\maketitle

\section{Introduction}


\section{Orignal Data}
\subsection{ITRF2008 IGS station}
The ITRF is The International Reference Frame, 
ITRF2008 is the new realization of the International Terrestrial Reference System. 
The ITRF2008 uses as input data time series of station positions and EOPs provided by the Technique Centers of the four space geodetic techniques (GPS, VLBI, SLR, DORIS). 
Based on completely reprocessed solutions of the four techniques, the ITRF2008 is expected to be an improved solution compared to ITF2005\upcite{ref1}.

In the file "ITRF2008\_GNSS.SSC.txt", we can find the coordinates of different stations at epoch 2005.0.
\begin{figure}[htbp]
    \centering
    \includegraphics[width=12cm]{./source/ITRF2008.png}
    \caption{ITRF2008\_GNSS.ssc.txt Description}
    \label{fig:ITRF2008}
\end{figure}

Time state: ?
\subsection{Station Obversations}
We were responsible for the computation of the positions and movements of three measurement stations: KIRU, MORP, and REYK. The locations are illustrated in the following figure:

An example of the observation file for each data set is provided below, including two time formats and XYZ coordinates.
\begin{figure}[htbp]
    \centering
    \includegraphics[width=12cm]{./source/xyz.png}
    \caption{station.xyz.txt Description}
    \label{fig:XYZ_obs}
\end{figure}
\subsection{NUVEL 1A Model}
NUVEL(Northeast University Velocity) is a the collective term for geophysical Earth models that describes observable
continental movements through a dynamic theory of plaet tectonics.

The "NNR\_NUVEL1A.txt" gives the rotation referred to epoch $t_0$. 
The file contains the following data, where the leftmost column represents the station name, 
and in that row, the angular velocity changes in three directions are provided (unit: radians per million years or $rad/My$).
\begin{figure}[htbp]
    \centering
    \includegraphics[width=12cm]{./source/xyz.png}
    \caption{NUVEL\-1A.txt Description}
    \label{fig:Nuvel-1A}
\end{figure}
\subsection{GIA Models}

\subsection{Other data}

\subsection{Matlab Code}

\section{Methodology}
\subsection{Transoformation to LHS}
[Geocentric cartesian coordinate system] A three-dimensional, earth-centered reference system in which locations are identified by their x, y, and z values. 
The x-axis is in the equatorial plane and intersects the prime meridian (usually Greenwich). 
The y-axis is also in the equatorial plane; it lies at right angles to the x-axis and intersects the 90-degree meridian. 
The z-axis coincides with the polar axis and is positive toward the north pole. The origin is located at the center of the sphere or spheroid.

[Local horizontal system] uses the Cartesian coordinates(East,Nort,Up) to represent position relative to a local origin. The local origin is described by the geodetic coordinates.

The initial coordinates are in the geocentric Cartesian coordinate system and need to be transformed into representation in the local horizontal coordinate system.
In this project, we use two angles and the ITRF2008 point positions as the original point, 

Calculate the angle according to stations' geodetic coordinates:

$$\lambda=\arctan\frac{y}{x}$$, $$\varphi=\arctan\frac{2}{\sqrt{x^{2}+y^{2}}}$$

Then we can get the rotation matrix:
$$R_2(\delta)=\begin{pmatrix}\cos\delta&0&-\sin\delta\\0&1&0\\\sin\delta&0&\cos\delta\end{pmatrix}\quad R_3(\delta)= \begin{pmatrix}\cos\delta&\sin\delta&0\\-\sin\delta&\cos\delta&0\\0&0&1\end{pmatrix}$$

Transformation of coordinates:
$$\left.\begin{pmatrix}x_{up}\\x_{east}\\x_{north}\end{pmatrix}=R_2(-\varphi^0)R_3(\lambda^0)\left(\begin{pmatrix}x_1\\x_2\\x_3\end{pmatrix}\right.-\begin{pmatrix}x_1^0\\x_2^0\\x_3^0\end{pmatrix}\right)$$
Velocities:
\subsection{Least Square Adjustment for Parameters Estimation}

\subsection{Model of Plate Tectonics}

\subsection{Program Description}

\section{Results and Analysis}
\subsection{Time Series and Linear Trend}

\subsection{Comparison of horizontal movements}

\subsection{Comparison of vertical movements}

\section{Conclusion}
\begin{thebibliography}{99}
    \bibitem{ref1}https://itrf.ign.fr/en/solutions/ITRF2008
    
\end{thebibliography}

\end{document}
